
\documentclass{beamer}
\usefonttheme[onlymath]{serif}

\usepackage[utf8]{inputenc}
\usepackage[russian]{babel}

\usepackage{graphicx}
\usepackage{ amssymb }
\usepackage{amsmath}

\newcommand{\norm}[1]{\left\lVert#1\right\rVert}

\usetheme{Madrid}

\title{Вариационные Автоенкодеры с мультимодальным prior'ом}
\author{Д. ~Мазур}

\institute[Гимназия 1505] 
{
  10 Б класс\\
  Школа 1505 "Преображенская"
 }

\begin{document}

    \begin{frame}
      \titlepage
    \end{frame}
    
    \begin{frame}{Моделируем сложные системы}{}
        Моделируем $p(x)$, где $x \in \mathcal{R}^D$ \pause

        $p(x) - ?$
    \end{frame}

    \begin{frame}{Вводим латентные переменные}{}
        $$z \in \mathcal{R}^d$$
    \end{frame}

    \begin{frame}{Кстати}{}
        $\mathbf{z}$ - приорное распределение

        $\mathbf{x}$ - постериорное распределение
    \end{frame}

    \begin{frame}{Решение}{}
        \setbeamertemplate{itemize items}[circle]
        \begin{itemize}
        \item $p_\theta(\mathbf{x}, \mathbf{z})$ - генеративная модель
            \begin{itemize}
               \item $p_\theta(\mathbf{x} \mid \mathbf{z})$ - декодер
               \item $p(\mathbf{z})$ - приорное распределение скрытых переменных
            \end{itemize}
        \item $q_\phi(\mathbf{z} \mid \mathbf{x})$ - Распределение с параметрами $\phi$
        \end{itemize}
    \end{frame}

    \begin{frame}{Выведение формулы}{}
        \begin{align*}
        \log p_\theta(\mathbf{z} \mid \mathbf{x}) = \mathcal{L}(\mathbf{x}, \theta, \phi) +  KL &\left(q_\phi(\mathbf{z} \mid \mathbf{x}) || p_\theta(\mathbf{z} \mid \mathbf{x})\right)
        \end{align*}
        \begin{align*}
        \mathcal{L} (\mathbf{x}; \theta, \phi) \approx  \log \frac{
      p_\theta (\mathbf{x}, \mathbf{z})
    }{
      q_\phi (\mathbf{z} \mid \mathbf{x})
    }, \qquad \mathbf{z} \sim q_\phi (\mathbf{z} \mid \mathbf{x}). 
        \end{align*}
    \end{frame}

    \begin{frame}{Оптимизируем}
       \begin{align*}
           \phi^*, \theta^* = \arg \max_{\phi \in \Phi, \theta \in \Theta} \mathcal{L}(\mathbf{x}, \theta, \phi)
       \end{align*} 
    \end{frame}

    \begin{frame}{Выбор $q_\phi$}
       Может быть не лучшим выбором для $q_\phi(\mathbf{z} \mid \mathbf{x})$
       \begin{align*}
           q_\phi(\mathbf{z} \mid \mathbf{x}) \sim \mathcal{N}(\mu, \sigma^2)
       \end{align*} 
    \end{frame}

    \begin{frame}{Решения}
        \begin{align*}
            q_\phi(\mathbf{z} \mid \mathbf{x}) \sim \frac{1}{N} \sum^{N}_{k=0} \mathcal{N}(\mu_k, \sigma_k^2)
            \pause
        \end{align*}

        \begin{align*}
            \sigma(p_\theta(\mathbf{z} \mid \mathbf{x}) \cdot \mathbf{W} + b)
        \end{align*}

        \begin{align*}
           \phi^*, \theta^* = \arg \max_{\phi \in \Phi, \theta \in \Theta} \mathcal{L}(\mathbf{x}, \theta, \phi) - \norm{W}
        \end{align*}
    \end{frame}

    \begin{frame}{Решения}
        \begin{thebibliography}{0}
            \bibitem{rezende}
            Variational Inference with Normalizing Flows
        \end{thebibliography}
    \end{frame}
   
    \begin{frame}{Библиография}
        \begin{thebibliography}{9}
            \bibitem{owen}
            Art B. Owen, Monte Carlo theory, methods and examples
            
            \bibitem{akosiorek} 
            Adam Kosiorek, What's wrong with VAEs

            \bibitem{kingma}
            Diederik P Kingma, Max Welling, Auto-Encoding Variational Bayes

            \bibitem{ejang}
            Eric Jang, Normalizing Flows tutorial
            
        \end{thebibliography}
    \end{frame}

\end{document}}